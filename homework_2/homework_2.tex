\documentclass{report}
\usepackage[showframe=false]{geometry}
\usepackage{titlesec}
\usepackage{amsmath}
\usepackage{graphicx}
\usepackage{qtree}
\usepackage{lscape}

\pagenumbering{gobble}

\geometry{tmargin=60pt,bmargin=90pt,lmargin=90pt,
rmargin=90pt}

\titleformat{\chapter}{\normalfont\huge}{\thechapter.}{20pt}{\huge}
\titlespacing*{\chapter} {0pt}{0pt}{10pt}

\begin{document}

\chapter{Section 1}
Give an unambiguous grammar to the language of arithmetic
expressions consisting of variable identifiers, integer
constants, float-point constants, addition (+), subtraction (-),
multiplication (*), division (/), and exponentiation (\#)
operators as well as parentheses. Adopt usual convention for
operator precedence and associativity. Note that (\#) has the
highest precedence and associates to the right. Assume that
id, integer and float are tokens.


Order of Operations

1. Parentheses

2. Power

3. Multiplication or division (Left - Right)

4. Addition or Subtration (Left - Right)

\begin{equation}
\begin{split}
  <addsubexp> & ::= <addsubexp> + <muldivexp> | <addsubexp> - <muldivexp> | <muldivexp> \\
  <muldivexp> & ::= <muldivexp> * <expexp> | <muldivexp> / <expexp> | <expexp> \\
  <expexp> & ::= <rootexp> \# <expexp> | <rootexp> \\
  <rootexp> & ::= (<addsubexp>) |  id  |  integer  |  float \\
\end{split}
\end{equation}

\chapter{Section 2}

Using the above grammar, draw a parse tree and an abstract
syntax tree for each of the following expressions where
English letters are variable identifiers.

\section{a}

$$X+Y*Z\#2$$

Parse Tree

\Tree [.$<addsubexp>$  
        [.$<addsubexp>$
          [.$<muldivexp>$
            [.$<expexp>$
              [.$<rootexp>$
                [.id
                [.X ]
                ]
              ]
            ]
          ]
        ]
        [.+ ]
        [.$<muldivexp>$
          [.$<muldivexp>$
            [.$<expexp>$
              [.$<rootexp>$
                [.id 
                  [.Y ]
                ]
              ]
            ]
          ]
          [.* ]
          [.$<expexp>$ 
            [.$<expexp>$ 
              [.$<rootexp>$ 
                [.id
                  [.Z ] 
                ]
              ]
            ]
            [.\# ]
            [.$<rootexp>$ 
              [.Integer
                [.2 ] 
              ]
            ]
          ]
        ]
      ]

Abstract Syntax

\Tree [.(+)
        [.X ]
        [.*
          [.Y ]
          [.\#
            [.Z ] 
            [.2 ] 
          ]
        ]
      ]

\newpage
\section{b}

$$X*(Y+7/2/V)$$

Parse Tree

\Tree [.$<addsubexp>$  
        [.$<muldivexp>$
          [.$<muldivexp>$
            [.$<expexp>$
              [.$<rootexp>$
                [.id
                  [.X ]
                ]
              ]
            ]
            !\qsetw{3cm}
          ]
          [.* !\qsetw{3cm} ]
          [.$<expexp>$
            [.$<rootexp>$
              [.( !\qsetw{3cm} ]
              [.$<addsubexp>$
                [.$<addsubexp>$
                  [.$<muldivexp>$
                    [.$<expexp>$
                      [.$<rootexp>$
                        [.id
                          [.Y ]
                        ]
                      ]
                    ]
                  ]
                  !\qsetw{5.5cm}
                ]
                [.+ !\qsetw{5.5cm} ]
                [.$<muldivexp>$
                  [.$<muldivexp>$
                    [.$<muldivexp>$
                      [.$<expexp>$
                        [.$<rootexp>$
                          [.Integer 
                            [.7 ]
                          ]
                        ]
                      ]
                    ]
                    [./ ]
                    [.$<expexp>$
                      [.$<rootexp>$
                        [.Integer 
                          [.2 ]
                        ]
                      ]
                    ]
                    !\qsetw{1cm}
                  ]
                  [./ !\qsetw{1cm} ]
                  [.$<expexp>$
                    [.$<rootexp>$
                      [.id
                        [.V ]
                      ]
                    ]
                    !\qsetw{1cm}
                  ]
                  !\qsetw{5.5cm}
                ]
                !\qsetw{3cm}
              ]
              [.) !\qsetw{3cm} ]
            ]
            !\qsetw{3cm}
          ]
        ]
      ]

Abstract Syntax

\Tree [.*
        [.X ]
        [.+
          [.Y ]
          [./ 
            [./ 
              [.7 ]
              [.2 ]
            ]
            [.V ]
          ]
          ]
      ]

\newpage
\section{c}
$$(X+Y)*(Z-X-Y)$$

Parse Tree

\Tree [.$<addsubexp>$  
        [.$<muldivexp>$
          [.$<muldivexp>$
            [.$<expexp>$
              [.$<rootexp>$
                [.( !\qsetw{3cm} ]
                [.$<addsubexp>$
                  [.$<addsubexp>$
                    [.$<muldivexp>$
                      [.$<expexp>$
                        [.$<rootexp>$
                          [.id
                            [.X ]
                          ]
                        ]
                      ]
                    ]
                  ]
                  [.+ ]
                  [.$<muldivexp>$
                    [.$<expexp>$
                      [.$<rootexp>$
                        [.id
                          [.Y ]
                        ]
                      ]
                    ]
                  ]
                  !\qsetw{3cm}
                ]
                [.) !\qsetw{3cm} ]
              ]
            ]
          ]
          [.* ]
          [.$<expexp>$
            [.$<rootexp>$
              [.( !\qsetw{4cm} ]
              [.$<addsubexp>$
                [.$<addsubexp>$
                  [.$<addsubexp>$
                    [.$<muldivexp>$
                      [.$<expexp>$
                        [.$<rootexp>$
                          [.id
                            [.Z ]
                          ]
                        ]
                      ]
                    ]
                  ]
                  [.- ]
                  [.$<muldivexp>$
                    [.$<expexp>$
                      [.$<rootexp>$
                        [.id
                          [.X ]
                        ]
                      ]
                    ]
                  ]
                  !\qsetw{4cm}
                ]
                [.- !\qsetw{4cm} ]
                [.$<muldivexp>$
                  [.$<expexp>$
                    [.$<rootexp>$
                      [.id
                        [.Y ]
                      ]
                    ]
                  ]
                  !\qsetw{3cm}
                ]
                !\qsetw{4cm}
              ]
              [.) !\qsetw{4cm} ]
            ]
          ]
        ]
      ]

Abstract Syntax

\Tree [.*
        [.+
          [.X ]
          [.Y ]
        ]
        [.- 
          [.- 
            [.Z ]
            [.X ]
          ]
          [.Y ]
        ]
      ]

\chapter{Section 3}
Extend the grammar in item 1 so that the language contains
Boolean expressions formed of comparison operators == and
$>=$ applied to arithmetic expressions and Boolean operators
\&\&, $||$ and ! (not) applied to Boolean expressions. The
operator ! (logical negation) is prefix and is of higher
precedence than \&\& (logical conjunction) that has higher
precedence than $||$ (logical disjunction).

Order of Operations

1. Parentheses

2. Power

3. Multiplication or division (Left - Right)

4. Addition or Subtration (Left - Right)

5. == or $>=$

6. !

7. \&\&

8. $||$

\begin{equation}
\begin{split}
  <logconj> & ::= <logconj> || <logdisj> | <logdisj> | <addsubexp> \\
  <logdisj> & ::= <logdisj> \&\& <logneg> | <logneg> \\
  <logneg> & ::= !<compopt> | <compopt> \\
  <compopt> & ::= <addsubexp> == <addsubexp> | <addsubexp> >= <addsubexp> \\
  <addsubexp> & ::= <addsubexp> + <muldivexp> | <addsubexp> - <muldivexp> | <muldivexp> \\
  <muldivexp> & ::= <muldivexp> * <expexp> | <muldivexp> / <expexp> | <expexp> \\
  <expexp> & ::= <rootexp> \# <expexp> | <rootexp> \\
  <rootexp> & ::= (<logconj>) |  id  |  integer  |  float \\
\end{split}
\end{equation}

\end{document}
